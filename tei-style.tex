\startxmlsetups xml:teisetups
        \xmlsetsetup{#1}{*}{-}
        \xmlsetsetup{#1}{teiHeader|fileDesc|titleStmt|title|author|persName|surname|forname}
        \xmlsetsetup{#1}{TEI|text|body|div|head|p|lb|cit|quote|bibl}{xml:*}
        \xmlsetsetup{#1}{note[@place='foot']}{xml:footnote}
% Sometimes a footnote is spread in two different note-tags.
% Eg. in schiller_erziehung02_1795.TEI-P5.xml

        \xmlsetsetup{#1}{pb}{xml:pb}
%        \xmlsetsetup{#1}{div[@n="2"]}{xml:subsection}
        \xmlsetsetup{#1}{hi[contains(@rendition, '\letterhash in')]}{xml:hi:in}
        \xmlsetsetup{#1}{hi[@rendition='\letterhash g']}{xml:hi:g}
        \xmlsetsetup{#1}{hi[contains(@rendition, '\letterhash et')]}{xml:hi:et}
        \xmlsetsetup{#1}{hi[contains(@rendition, '\letterhash aq')]}{xml:hi:aq}
        \xmlsetsetup{#1}{hi[contains(@rendition, '\letterhash k')]}{xml:hi:k}
        \xmlsetsetup{#1}{p[contains(@rendition, '\letterhash c')]}{xml:p:c}
        \xmlsetsetup{#1}{bibl[contains(@rendition, '\letterhash k')]}{xml:bibl:k}

% I have to setup the highlighting patterns for each tag. This is bad.
% A mechanism to match highlighting patterns independently would be
% better.

\stopxmlsetups

\xmlregistersetup{xml:teisetups}

\startxmlsetups xml:teiHeader
	\xmlflush{#1}
\stopxmlsetups

\startxmlsetups xml:fileDesc
	\xmlflush{#1}
\stopxmlsetups

\startxmlsetups xml:titleStmt
	\xmlflush{#1}
\stopxmlsetups

\startxmlsetups xml:title
  \setupinteraction[title={\xmlflush{#1}}]
	\title{\xmlflush{#1}}
\stopxmlsetups

\startxmlsetups xml:author
	\xmlflush{#1}
\stopxmlsetups

\startxmlsetups xml:persName
	\xmlflush{#1}
\stopxmlsetups

\startxmlsetups xml:forname
	\xmlflush{#1}
\stopxmlsetups


\startxmlsetups xml:surname
	\xmlflush{#1}
\stopxmlsetups

\startxmlsetups xml:TEI
	\xmlflush{#1}
\stopxmlsetups

\startxmlsetups xml:text
	\xmlflush{#1}
\stopxmlsetups

\startxmlsetups xml:body
%  \startlinenumbering % Does not start correctly on the first page
	\xmlflush{#1}
\stopxmlsetups

\setuplinenumbering[location=inner,
                    step=5,
                    method=page,
                    style=\tfxx,
                    align=left,
                    distance=0.3em,
                    width=0.3cm]

\startxmlsetups xml:div
	\xmlflush{#1}
\stopxmlsetups

\setuphead[section]
  [number=no]

\startxmlsetups xml:head
  {\startsection
  \startalign[center]
  \tfc
  \xmlflush{#1}
  \stopalign
  \stopsection}
\stopxmlsetups

\startxmlsetups xml:hi:in
	{\placeinitial \xmlflush{#1}}
\stopxmlsetups

\startxmlsetups xml:hi:g
	{\em \xmlflush{#1}}
\stopxmlsetups

\startxmlsetups xml:hi:et
	{\em \xmlflush{#1}}
\stopxmlsetups

\startxmlsetups xml:hi:aq
	{\ss \xmlflush{#1}}
\stopxmlsetups

\startxmlsetups xml:hi:k
	\cap{\xmlflush{#1}}
\stopxmlsetups

\startxmlsetups xml:hi:c
	{\startalign[center]
  \xmlflush{#1}
  \stopalign}
\stopxmlsetups

\startxmlsetups xml:cit
	\xmlflush{#1}
\stopxmlsetups

\startxmlsetups xml:quote
	\xmlflush{#1}
\stopxmlsetups

\startxmlsetups xml:bibl
  \rightaligned{
	\xmlflush{#1}}
\stopxmlsetups

\startxmlsetups xml:bibl:k
  \rightaligned{
	\cap{\xmlflush{#1}}}
\stopxmlsetups


\startxmlsetups xml:footnote
	\footnote{\xmlflush{#1}}
\stopxmlsetups

\startxmlsetups xml:p
  \par
	\xmlflush{#1}
  \par
\stopxmlsetups

\startxmlsetups xml:p:c
	{\par
  \startalign[center]
  \xmlflush{#1}
  \stopalign
  }
\stopxmlsetups

\startxmlsetups xml:cit:quote:hi
  \startblockquote
	\xmlflush{#1}
  \stopblockquote
\stopxmlsetups


\startxmlsetups xml:pb
	{\tfxx/\inothermargin{\xmlatt{#1}{n}}}
\stopxmlsetups


% This mode typesets the source with the original
% linebreaks and hyphenation of the scanned book

\startmode[original]
\startxmlsetups xml:lb
	\crlf
\stopxmlsetups
\stopmode

% This mode typesetzs the source in a new modern fashion.
% Original linebreaks and hyphenations of the scanned book
% are removed

\startmode[newedition]
\startxmlsetups xml:p
  \par
	\cldcontext{string.gsub([[\xmlflush{#1}]], "¬", "\\nospace")}
  \par
\stopxmlsetups

\startxmlsetups xml:lb
\stopxmlsetups
\stopmode

\setupinitial[font=Bold sa 4,distance=3pt,state=start,n=3]


% We need the font FreeSerif to display combining diacritical marks.
% Maybe this should be moved to a style environment

\definefontfamily
    [mainface]
    [rm]
    [FreeSerif]
%    [UnifrakturMaguntia] % If we have sources that needs Fraktur

\setupbodyfont
    [mainface, 12pt]



\placebookmarks
    [part,chapter,hiddentitle,section,subsection,subsubsection]
    [part,chapter,hiddentitle]
