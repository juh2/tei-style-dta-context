\startxmlsetups xml:teisetups
    \xmlsetsetup{#1}{*}{-}
    \xmlsetsetup{#1}{teiHeader|fileDesc|titleStmt|title|author|persName|surname|forname|editionStmt|publicationStmt|encodingDesc|profileDesc|langUsage|language}{xml:*}
    \xmlsetsetup{#1}{TEI|text|body|div|p|lb|cit|quote|bibl}{xml:*}
    \xmlsetsetup{#1}{div[@n='1']/head}{xml:chapter}
    \xmlsetsetup{#1}{div[@n='2']/head}{xml:section}
    \xmlsetsetup{#1}{fileDesc/titleStmt}{} % remove unwanted titles
    \xmlsetsetup{#1}{div/lb}{} % remove linebreak outside titles
    \xmlsetsetup{#1}{p/lb}{} % remove linebreak inside paragraphs
    \xmlsetsetup{#1}{note[@place='foot']}{xml:footnote}
% Sometimes a footnote is spread in two different note-tags.
% Eg. in schiller_erziehung02_1795.TEI-P5.xml

    \xmlsetsetup{#1}{pb}{xml:pb}
    \xmlsetsetup{#1}{hi[contains(@rendition, '\letterhash in')]}{xml:hi:in}
    \xmlsetsetup{#1}{hi[@rendition='\letterhash g']}{xml:hi:g}
    \xmlsetsetup{#1}{head/hi[@rendition='\letterhash g']}{xml:normalflush}
    \xmlsetsetup{#1}{hi[contains(@rendition, '\letterhash et')]}{xml:hi:et}
    \xmlsetsetup{#1}{hi[contains(@rendition, '\letterhash aq')]}{xml:hi:aq}
    \xmlsetsetup{#1}{hi[contains(@rendition, '\letterhash k')]}{xml:hi:k}
    \xmlsetsetup{#1}{p[contains(@rendition, '\letterhash c')]}{xml:p:c}
    \xmlsetsetup{#1}{bibl[contains(@rendition, '\letterhash k')]}{xml:bibl:k}

% I have to setup the highlighting patterns for each tag. This is bad.
% A mechanism to match highlighting patterns independently would be
% better.

\stopxmlsetups

\xmlregistersetup{xml:teisetups}

\startxmlsetups xml:teiHeader
    \xmlflush{#1}
\stopxmlsetups

\startxmlsetups xml:fileDesc
    \xmlflush{#1}
\stopxmlsetups

\startxmlsetups xml:titleStmt
    \xmlflush{#1}
\stopxmlsetups

\startxmlsetups xml:title
  \setupinteraction[title={\xmlflush{#1}}]
   \title{\xmlflush{#1}}
\stopxmlsetups

\startxmlsetups xml:author
    \xmlflush{#1}
\stopxmlsetups

\startxmlsetups xml:persName
    \xmlflush{#1}
\stopxmlsetups

\startxmlsetups xml:forname
    \xmlflush{#1}
\stopxmlsetups

\startxmlsetups xml:surname
    \xmlflush{#1}
\stopxmlsetups

\startxmlsetups xml:editionStmt
    \xmlflush{#1}
\stopxmlsetups

\startxmlsetups xml:publicationStmt
    \xmlflush{#1}
\stopxmlsetups

\startxmlsetups xml:encodingDesc
    \xmlflush{#1}
\stopxmlsetups

\startxmlsetups xml:profileDesc
    \xmlflush{#1}
\stopxmlsetups

\startxmlsetups xml:langUsage
    \xmlflush{#1}
\stopxmlsetups

\startxmlsetups xml:language
    \mainlanguage[\xmlatt{#1}{ident}]
\stopxmlsetups

\installlanguage[deu][de]

\startxmlsetups xml:TEI
    \xmlflush{#1}
\stopxmlsetups

\startxmlsetups xml:text
    \xmlflush{#1}
\stopxmlsetups

\startxmlsetups xml:body
%  \startlinenumbering % Does not start correctly on the first page
    \xmlflush{#1}
\stopxmlsetups

\setuplinenumbering[location=inner,
            step=5,
            method=page,
            style=\tfxx,
            align=left,
            distance=0.3em,
            width=0.3cm]

\startxmlsetups xml:div
    \xmlflush{#1}
\stopxmlsetups

\setuphead[chapter]
  [number=no,
   align=center,
   style=\tfd,
   page=,]

\setuphead[section]
  [number=no,
   align=center,
   style=\itc]

\startxmlsetups xml:lb
    \crlf
\stopxmlsetups

\startxmlsetups xml:normalflush
  \xmlflush{#1}
\stopxmlsetups

\startxmlsetups xml:chapter
  \startchapter[title={\xmlflush{#1}}]\stopchapter
\stopxmlsetups

\startxmlsetups xml:section
  \startsection[title={\xmlflush{#1}}]\stopsection
\stopxmlsetups

\startxmlsetups xml:hi:in
    {\placeinitial \xmlflush{#1}}
\stopxmlsetups

\startxmlsetups xml:hi:g
    {\em \xmlflush{#1}}
\stopxmlsetups

\startxmlsetups xml:hi:et
    {\em \xmlflush{#1}}
\stopxmlsetups

\startxmlsetups xml:hi:aq
    {\ss \xmlflush{#1}}
\stopxmlsetups

\startxmlsetups xml:hi:k
    \cap{\xmlflush{#1}}
\stopxmlsetups

\startxmlsetups xml:hi:c
    {\startalign[center]
  \xmlflush{#1}
  \stopalign}
\stopxmlsetups

\startxmlsetups xml:cit
    \xmlflush{#1}
\stopxmlsetups

\startxmlsetups xml:quote
    \xmlflush{#1}
\stopxmlsetups

\startxmlsetups xml:bibl
  \rightaligned{
    \xmlflush{#1}}
\stopxmlsetups

\startxmlsetups xml:bibl:k
  \rightaligned{
    \cap{\xmlflush{#1}}}
\stopxmlsetups


\startxmlsetups xml:footnote
    \startfootnote\xmlflush{#1}\stopfootnote
\stopxmlsetups

\startxmlsetups xml:p
  \par
    \xmlflush{#1}
  \par
\stopxmlsetups

\startxmlsetups xml:p:c
    {\par
  \startalign[center]
  \xmlflush{#1}
  \stopalign
  }
\stopxmlsetups

\startxmlsetups xml:cit:quote:hi
  \startblockquote
    \xmlflush{#1}
  \stopblockquote
\stopxmlsetups

\startxmlsetups xml:pb
    {\tfxx/\inothermargin{\xmlatt{#1}{n}}}
\stopxmlsetups


% This mode typesets the source with the original
% linebreaks and hyphenation of the scanned book

\startmode[original]
\startxmlsetups xml:lb
    \crlf
\stopxmlsetups
\stopmode

% This mode typesets the source in a new modern fashion.
% Original linebreaks and hyphenations of the scanned book
% are removed

\startmode[newedition]
\startxmlsetups xml:p
  \par
    \cldcontext{string.gsub([[\xmlflush{#1}]], "¬", "\\nospace")}
  \par
\stopxmlsetups

\startxmlsetups xml:lb
    \crlf
\stopxmlsetups
\stopmode

\setupinitial[font=Bold sa 4,distance=3pt,state=start,n=3]

% We need the font FreeSerif to display combining diacritical marks.
% Maybe this should be moved to a style environment

\definefontfamily
    [mainface]
    [rm]
    [FreeSerif]
%   [UnifrakturMaguntia] % If we have sources that needs Fraktur

\setupbodyfont
    [mainface, 12pt]

\setupinteraction
    [state=start,
     color=,
     style=,]

\placebookmarks
    [part,chapter,section,subsection,subsubsection]
    [part,chapter]
